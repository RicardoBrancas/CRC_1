\documentclass[9pt,a4paper,twocolumn]{article}
\usepackage[margin=0.9in]{geometry}
\usepackage{hyperref}
\usepackage[final]{pdfpages}
\usepackage{subcaption}
\usepackage{cite}
\usepackage[acronym]{glossaries}

 \setlength{\columnsep}{0.25in}

\makeglossaries

\newacronym{scc}{SCC}{Strongly Connected Components}

\title{An Analysis of the Portuguese Section of Wikipedia}

\author{Daniel Ramos \\ 81620 \and Miguel Tavares \\ 83528 \and Ricardo Brancas  \\ 83557}

\begin{document}
\maketitle

\section{Introduction}
The goal of this project is to analyse and characterise a real world network, and to  get 

acquainted with tools and methods which will be useful in following endeavours.
As such, we have chosen to analyse a snapshot of the Portuguese section of Wikipedia \cite{dataset};
we have chosen such a large network in order to get familiarized with tools such as Webgraph \cite{webgraph}.

%TODO

To analyse our data we use Webgraph, including snippets of code made available
by Prof. Alexandre Francisco \cite{aplf}.

\section{Methods}
The network was originally represented as an edge list which was subsequently converted in an adjacency list (ASCII Graph), by use of a simple \texttt{C++} program, as this is the input
format for Webgraph's compression algorithms.

After running Webgraph, we used a Python script to parse the output and generate the figures present in this document.


\section{Results}

The network we have chosen contains $1\,603\,222$ vertices and $49\,021\,409$ edges. Both the minimum in and out-degrees are $0$, the maximum in-degree is $207\,254$ while the maximum out-degree is $12\,237$, finally the combined average degree is approximately $61.15$. Furthermore, there are 121 \acrlong{scc}, the largest being comprised of $1\,602\,960$ nodes.
\vspace{1\baselineskip}

In figures ~\ref{fig:inddist} and~\ref{fig:outddist} we present the cumulative in and out, respectively, degree distributions and their approximate power law regression. The regression was obtained using a simplified method based on the one described by Clauset et al. \cite{Clauset2009}.

\begin{figure}[h]
	\centering
	\begin{subfigure}{.475\textwidth}
		\centering
		\includegraphics[width=\linewidth]{wikipedia_pt_in.pdf}
		\caption{Cumulative in-degree distribution.}
		\label{fig:inddist}
	\end{subfigure}
	\begin{subfigure}{.475\textwidth}
		\centering
		\includegraphics[width=\linewidth]{wikipedia_pt_out.pdf}
		\caption{Cumulative out-degree distribution.}
		\label{fig:outddist}
	\end{subfigure}
	\caption{Degree distributions.}
\end{figure}

Figure \ref{fig:sccdist} shows the number of \acrshort{scc} with respect to their cardinality. We notice a giant strongly connected component, as would be expected.

\begin{figure}[h]
	\centering
	\includegraphics[width=\linewidth]{wikipedia_pt_sccdistr.pdf}
	\caption{Strongly Connected Component distribution.}
	\label{fig:sccdist}
\end{figure}

In figure \ref{fig:neighfun} is represented the approximate neighbourhood function as computed by Webgraph. This measure represents for each $t \in N$, the number of pairs of nodes $ \langle x, y \rangle $ such that $y$ is reachable from $x$ in less than $t$ hops \cite{Boldi2011HyperANFAT}.

\begin{figure}[h]
	\centering
	\includegraphics[width=\linewidth]{wikipedia_pt_neighbourhood_function.pdf}
	\caption{Approximate neighbourhood function.}
	\label{fig:neighfun}
\end{figure}

We have also obtained an estimation of the average path length through HyperBall \cite{Boldi2011HyperANFAT}. The estimated value is $ 4.9290 \pm 0.0026 $. This result reinforces what has been previously discussed, as it can be seen in Figure \ref{fig:neighfun} roughly a half of the nodes can be reached from each other in just 5 hops.

\section{Discussion}

The cumulative in-degree $\gamma = 2.42$ parameter is lower than its out-degree counterpart $\gamma = 3.89$. Although our data is unlabelled, we believe this phenomenon occurs because there are few Wikipedia pages which reference most of the others (ie. indexes), leading to a higher $\gamma$ in the out-degree distribution. Conversely, the in-degree distribution parameter $\gamma = 2.42$ is lower because pages are comparatively more uniformly referenced.

In what regards the \acrshort{scc}s, as previously mentioned, we have found a giant one %ugly
containing almost all the nodes ($1\,602\,960$) in the graph, and also a very few small ones. These results are in accordance with the average degrees ($> ln(N)$) obtained, which imply a connected regime.

By analysing the approximate neighbourhood function, we have concluded that at around 10 hops, all the reachable nodes become connected.


\printglossary[type=\acronymtype]

\bibliographystyle{plain}
\bibliography{references}

\end{document}
